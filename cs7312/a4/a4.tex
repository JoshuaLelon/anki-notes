% -*- coding:utf-8 -*-
% LATEX PREAMBLE --- needs to be imported manually
\documentclass[12pt]{article}
% \special{papersize=3in,5in}

\usepackage[utf8]{inputenc}
\usepackage{amssymb,amsmath,amsthm}
\pagestyle{empty}
\setlength{\parindent}{0in}

\newcommand{\detail}[1]{{\LARGE#1\par}~}
\newcommand{\refs}[1]{{\LARGE\textit{References: }#1\par}\hfill.}
\newcommand*{\abs}[1]{\left\vert#1\right\vert}
\newcommand{\lgmth}[1]{\begingroup\LARGE\[#1\]\endgroup}

%%% commands that do not need to imported into Anki:
\usepackage{mdframed}
\newcommand*{\tags}[1]{\paragraph{tags: }#1\bigskip}
\newcommand*{\xfield}[1]{\begin{mdframed}[font=\sffamily\LARGE]\centering #1\end{mdframed}\bigskip}
\newenvironment{field}{}{}
\newcommand*{\xplain}[1]{\begin{mdframed}\texttt{#1}\end{mdframed}\bigskip}
\newenvironment{plain}{\ttfamily}{\par}
\newenvironment{note}{}{}
% END OF THE PREAMBLE

% Integral dx
\newcommand{\dx}{\mathrm{d}x}

%
% Various Helper Commands
%

% For derivatives
\newcommand{\deriv}[1]{\frac{\mathrm{d}}{\mathrm{d}x} (#1)}

% For partial derivatives
\newcommand{\pderiv}[2]{\frac{\partial}{\partial #1} (#2)}


% Alias for the Solution section header
\newcommand{\solution}{\textbf{\large Solution}}

% Probability commands: Expectation, Variance, Covariance, Bias
\newcommand{\E}{\mathrm{E}}
\newcommand{\Var}{\mathrm{Var}}
\newcommand{\Cov}{\mathrm{Cov}}
\newcommand{\Bias}{\mathrm{Bias}}

% Formatting commands:

\newcommand{\mt}[1]{\ensuremath{#1}}
\newcommand{\nm}[1]{\textrm{#1}}

\newcommand\bsc[2][\DefaultOpt]{%
  \def\DefaultOpt{#2}%
  \section[#1]{#2}%
}
\newcommand\ssc[2][\DefaultOpt]{%
  \def\DefaultOpt{#2}%
  \subsection[#1]{#2}%
}
\newcommand{\bgpf}{\begin{proof} $ $\newline}

\newcommand{\bgeq}{\begin{equation*}}
\newcommand{\eeq}{\end{equation*}}	

\newcommand{\balist}{\begin{enumerate}[label=\alph*.]}
\newcommand{\elist}{\end{enumerate}}

\newcommand{\bilist}{\begin{enumerate}[label=\roman*)]}	

\newcommand{\bgsp}{\begin{split}}
% \newcommand{\esp}{\end{split}} % doesn't work for some reason.

\newcommand\prs[1]{~~~\textbf{(#1)}}

\newcommand{\lt}[1]{\textbf{Let: } #1}
     							   %  if you're setting it to be true
\newcommand{\supp}[1]{\textbf{Suppose: } #1}
     							   %  Suppose (if it'll end up false)
\newcommand{\wts}[1]{\textbf{Want to show: } #1}
     							   %  Want to show
\newcommand{\as}[1]{\textbf{Assume: } #1}
     							   %  if you think it follows from truth
\newcommand{\bpth}[1]{\textbf{(#1)}}

\newcommand{\step}[2]{\begin{equation}\tag{#2}#1\end{equation}}
\newcommand{\epf}{\end{proof}}

\newcommand{\dbs}[3]{\mt{#1_{#2_#3}}}

\newcommand{\sidenote}[1]{-----------------------------------------------------------------Side Note----------------------------------------------------------------
#1 \

---------------------------------------------------------------------------------------------------------------------------------------------}

% Analysis / Logical commands:

\newcommand{\br}{\mt{\mathbb{R}} }       % |R
\newcommand{\bq}{\mt{\mathbb{Q}} }       % |Q
\newcommand{\bn}{\mt{\mathbb{N}} }       % |N
\newcommand{\bc}{\mt{\mathbb{C}} }       % |C
\newcommand{\bz}{\mt{\mathbb{Z}} }       % |Z
\newcommand{\bi}{\bnm{\mathbb{R}}{\mathbb{Q}}} % |Irrationals 

\newcommand{\ep}{\mt{\epsilon} }         % epsilon
\newcommand{\fa}{\mt{\forall} }          % for all
\newcommand{\afa}{\mt{\alpha} }
\newcommand{\bta}{\mt{\beta} }
\newcommand{\dta}{\mt{\delta} }
\newcommand{\mem}{\mt{\in} }
\newcommand{\exs}{\mt{\exists} }

\newcommand{\es}{\mt{\emptyset} }        % empty set
\newcommand{\sbs}{\mt{\subset} }         % subset of
\newcommand{\fs}[2]{\{\uw{#1}{1}, \uw{#1}{2}, ... \uw{#1}{#2}\}}

\newcommand{\lra}{ \mt{\longrightarrow} } % implies ----->
\newcommand{\rar}{ \mt{\Rightarrow} }     % implies -->
\newcommand{\lba}{ \mt{\longmapsto} }     % element maps to |--->

\newcommand{\lla}{ \mt{\longleftarrow} }  % implies <-----
\newcommand{\lar}{ \mt{\Leftarrow} }      % implies <--

\newcommand{\av}[1]{\mt{|}#1\mt{|}}  % absolute value

\newcommand{\prn}[1]{(#1)}
\newcommand{\bk}[1]{\{#1\}}
\newcommand{\abk}[1]{\mt{\langle}#1\mt{\rangle}}

\newcommand{\ps}{\mt{+} }
\newcommand{\ms}{\mt{-} }

\newcommand{\ls}{\mt{<} }
\newcommand{\gr}{\mt{>} }

\newcommand{\lse}{\mt{\leq} }
\newcommand{\gre}{\mt{\geq} }

\newcommand{\eql}{\mt{=} }

\newcommand{\pr}{\mt{^\prime} } 		   % prime (i.e. R')
\newcommand{\uw}[2]{#1\mt{_{#2}}}
\newcommand{\uf}[2]{#1\mt{^{#2}}}
\newcommand{\frc}[2]{\mt{\frac{#1}{#2}}}
\newcommand{\lmti}[1]{\mt{\displaystyle{\lim_{#1 \to \infty}}}}
\newcommand{\limt}[2]{\mt{\displaystyle{\lim_{#1 \to #2}}}}

\newcommand{\bnm}[2]{\mt{#1\setminus{#2}}}
\newcommand{\bnt}[2]{\mt{\textrm{#1}\setminus{\textrm{#2}}}}

\newcommand{\urng}[2]{\mt{\bigcup_{#1}^{#2}}}
\newcommand{\nrng}[2]{\mt{\bigcap_{#1}^{#2}}}
\newcommand{\nck}[2]{\mt{{#1 \choose #2}}}

     							   
\newcommand{\eqn}[1]{\[#1\]}
\newcommand{\splt}[1]{\begin{split}#1\end{split}}

\newcommand{\infy}{\mt{\infty} }
\newcommand{\unn}{\mt{\cup} }
\newcommand{\inn}{\mt{\cap} }
\newcommand\tab[1][1cm]{\hspace*{#1}}
\newcommand{\rln}{ \mt{\sim} }
\newcommand{\dvd}{ \mt{\vert} }
\newcommand{\ndvd}{ \mt{\not\vert} }
\newcommand{\eqw}{ \mt{ \equiv } }
\newcommand{\lcg}{ \mt{\gamma} }

\newcommand{\edp}{\mt{\bigoplus} }

\newcommand{\wit}[1]{\mt{\widetilde{#1}}}

\newcommand{\mxc}[5]{ % Matrix Column: entry entry entry entry DIMENSION
\underset{#5 \times 1}{
  \begin{bmatrix}
     #1 \\
     #2 \\
     #3 \\
     #4
  \end{bmatrix}
  }
}

\newcommand{\mxr}[5]{ % Matrix Row:    entry entry entry entry DIMENSION
\underset{1 \times #5}{
  \begin{bmatrix}
     #1 & #2 & #3 & #4
  \end{bmatrix}
  }
}

\begin{document}

Joshua Mitchell

CS 7312 Assignment 4

\

\begin{verbatim}
  1.1 (5 points) Draw the inverted index that would be built for the following
  document collection. (See Figure 1.3, Page 6 of the textbook, for an example.)

Doc 1 	new home sales top forecasts

Doc 2 	home sales rise in july

Doc 3 	increase in home sales in july

Doc 4 	july new home sales rise
\end{verbatim}


\

\begin{tabular}{l|cc}
  Value & Key& \\
  \hline
  new & 1, 4& \\
  home & 1, 2, 3, 4 & \\
  sales & 1, 2, 3, 4 & \\
  top & 1 & \\
  forecasts & 1 & \\
  rise & 2, 4 & \\
  in & 2, 3 & \\
  july & 2, 3, 4 & \\
  increase & 3 &
\end{tabular}

\begin{verbatim}
  1.2 (5 points) Consider the following fragment of a positional index
  with the format: 

word: document: <position, position, . . .>; document: < position, . . .> 
. . .
Gates: 1: <3>; 2: <6>; 3: <2,17>; 4: <1>;
IBM: 4: <3>; 7: <14>;
Microsoft: 1: <1>; 2: <1,21>; 3: <3>; 5: <16,22,51>;

The /k operator, word1 /k word2 finds occurrences of word1 within k words of word2 (on either side), where k is a positive integer argument. Thus k = 1 demands that word1 be adjacent to word2.

Describe the set of documents that satisfy the query Gates /2 Microsoft.
\end{verbatim}

\

Documents 1 and 3 satisfy this query.

Gates is within 2 words of Microsoft in document 1, and Gates is within 1 word of Microsoft in document 3.

\

\begin{verbatim}
  (20 points) Computing ranking scores in a search engine with the lnc.ltc 
  weighting scheme. Let the query be "good student” and the document be
  "good bad student good bad instructor”. Fill out the empty columns in the
  following table and then compute the cosine similarity between the query 
  vector and the document vector. In the table, df denotes document frequency,
  idf denotes inverse document frequency (i.e., idft = log10N/dft),
  tf denotes term frequency, log tf denotes the tf weight based on log-frequency 
  weighting as shown in slides (i.e., 1+log10tft,d for tft,d> 0 and 0 otherwise), 
  q is the query vector, q’ is the length-normalized q, d is the document vector, 
  and d’ is the length-normalized d.  Assume N = 10,000,000.
  
  The cosine similarity between q and d is the dot product of q’ and d’, which is:                             
\end{verbatim}

\begin{tabular}{lcc|cccc|cccc}
  & & & query & & & & document & & & \\
  \hline
  terms & df & idf & tf & log tf & q & q' & tf & log tf & d & d' \\
  \hline
  bad & 1000 & 4 & 0 & 0 & 0 & 0 & 2 & 1.3 & 1.3 & 0.561 \\
  good & 10000 & 3 & 1 & 1 & 3 & 0.794 & 2 & 1.3 & 1.3 & 0.561 \\
  instructor & 10 & 6 & 0 & 0 & 0 & 0 & 1 & 1 & 1 & 0.431 \\
  student & 50000 & 2.3 & 1 & 1 & 2.3 & 0.608 & 1 & 1 & 1 & 0.431 \\
   
\end{tabular}

\

Dot product of q' and d': 0.707

\

\begin{verbatim}
  3.1 (10 points) An IR system returns 8 relevant documents, and 10 nonrelevant
  documents. There are a total of 20 relevant documents in the collection.
  What is the precision of the system for this search, what is its recall?
  what is the balanced F measure?
\end{verbatim}

\

The precision is 0.44, the recall is 0.4, and the balanced F measure is 0.421.

\

\begin{verbatim}
  3.2 (10 points) Consider an information need for which there are 4 relevant
  documents in the collection. Compare two systems that run on this collection. 
  Their top 10 results are judged for relevance as follows (the leftmost item 
  is the top ranked search result):

System 1: R N R N N N N N R R

System 2: N R N N R R R N N N

a. What is the MAP of each system? Which has a higher MAP?
b. What is the R-precision of each system? Does it rank the systems the 
same as MAP?
\end{verbatim}

\

a.

System 1 MAP: (1 + 0 + 2/3 + 0 + 0 + 0 + 0 + 0 + 3/9 + 4/10) / 4 = 0.6

System 2 MAP: (0 + 1/2 + 0 + 0 + 2/5 + 3/6 + 4/7 + 0 + 0 + 0) / 4 = 0.493

System 1 has the higher MAP.

\

b.

The R precision of System 1 is 0.5, and the R precision of System 2 is 0.25.

The ranking order of each system is the same (System 1 comes first).

\


\begin{verbatim}
  3.3 (20 points) The following list of R’s and N’s represents relevant (R)
   and nonrelevant (N) documents in a ranked list of 20 documents in
   response to a query from a collection of 10,000 documents. The 
   leftmost item is the top ranked search result. This list shows 6
   relevant documents. Assume that there are 8 relevant documents in
   total in the collection.

R R N N N N N N R N R N N N R N N N N R

a. What is the precision of the system on the top 20?
b. What is the F1 (balanced F measure) on the top 20?
c. What is the uninterpolated precision of the system at 25% recall?
d. What is the interpolated precision at 33% recall?
e. Assume that these 20 documents are the complete result set of the system.
What is the MAP for the query?

Assume, now, instead, that the system returned the entire 10,000 documents
in a ranked list, and these are the first 20 results returned.

f. What is the largest possible MAP that this system could have?
g. What is the smallest possible MAP that this system could have?
\end{verbatim}

a. Precision: 6/20 = 0.3

b. F1: (2 * 0.3 * 0.75) / (0.3 + 0.75) = 0.429

c. The precision values at 25\% recall are: 2/2, 2/3, 2/4, 2/5, 2/6, 2/7, and 2/8. The average of these values is 0.491

d. The highest precision (0.364) such that recall \gre 33\% is at 11 results selected from the collection.

\

e. System MAP: (1 + 1 + 0 + 0 + 0 + 0 + 0 + 0 + 3/9 + 0 + 4/11 + 0 + 0 + 0 + 5/15 + 0 + 0 + 0 + 0 + 6/20) / 6 = 0.555

\

f. Largest MAP: (1 + 1 + 0 + 0 + 0 + 0 + 0 + 0 + 3/9 + 0 + 4/11 + 0 + 0 + 0 + 5/15 + 0 + 0 + 0 + 0 + 6/20 + 7/21 + 8/22) / 8 = 0.503

\

g. Smallest MAP: (1 + 1 + 0 + 0 + 0 + 0 + 0 + 0 + 3/9 + 0 + 4/11 + 0 + 0 + 0 + 5/15 + 0 + 0 + 0 + 0 + 6/20 + 0 + ... + 0 + 7/9999 + 8/10000) / 8 = 0.416

\

\begin{verbatim}
  4.1 (15 points) Consider a web graph with three nodes 1, 2 and 3. The links
  are as follows: 1 -> 2, 3 -> 2, 2 -> 1, 2 -> 3. Write down the transition
  probability matrices for the random surfer’s walk with teleporting, for
  the following three values of the teleport probability:
  (a) a = 0; (b) a = 0.5 and (c) a = 1.
  
\end{verbatim}

\begin{displaymath} a:
  \begin{bmatrix}
     0 & 1 & 0\\
     1/2 & 0 & 1/2 \\
     0 & 1 & 0  
  \end{bmatrix} b:
  \begin{bmatrix}
     1/6 & 2/3 & 1/6 \\
     5/12 & 1/6 & 5/12 \\
     1/6 & 2/3 & 1/6 
  \end{bmatrix} c:
  \begin{bmatrix}
     1/3 & 1/3 & 1/3 \\
     1/3 & 1/3 & 1/3 \\
     1/3 & 1/3 & 1/3  
  \end{bmatrix}
\end{displaymath}

\

\

\

\begin{verbatim}
  4.2 (15 points) For the web graph shown below,
  compute PageRank, hub and authority scores for each of the three pages.
  
  PageRank: Assume that at each step of the PageRank random walk, 
  we teleport to a random page with probability 0.1, with a uniform 
  distribution over which particular page we teleport to.

  Hubs/Authorities: Normalize the hub (authority) scores so that the
  maximum hub (authority) score is 1.
\end{verbatim}

\begin{displaymath}
    P = \begin{bmatrix}
     1/30 & 29/60 & 29/60 \\
     1/30 & 1/30 & 14/15 \\
     1/30 & 14/15 & 1/30  
  \end{bmatrix}
\end{displaymath}

The above is the normalized random walk matrix for the web graph. It turns out that (1/3, 1/3, 1/3)$^T$ is an eigenvector with an eigenvalue of 1.

\

(I picked (1/3, 1/3, 1/3) initially like the example in the slides, and it converged immediately.)

\

 Thus, the PageRank of each vertex is 1/3.
 
 \
 
 For HITS, the initial h(x) and a(x) scores as well as the updates are as follows:
 
 \
 
\begin{tabular}{l|ccc}
  Vertex & h(x) & a(x) \\
  \hline
  1 & 2 & 0 \\
  2 & 1 & 2 \\
  3 & 1 & 2
\end{tabular} \rar 
\begin{tabular}{l|ccc}
  Vertex & h(x) & a(x) \\
  \hline
  1 & 1 & 0 \\
  2 & 1/2 & 1 \\
  3 & 1/2 & 1
\end{tabular} \rar
\begin{tabular}{l|ccc}
  Vertex & h(x) & a(x) \\
  \hline
  1 & 1 & 0 \\
  2 & 1/2 & 3/4 \\
  3 & 1/2 & 3/4
\end{tabular}

\

So it appears that vertex q1 is the best hub, and q2/q3 are the best authorities.


\end{document}