% -*- coding:utf-8 -*-
% LATEX PREAMBLE --- needs to be imported manually
\documentclass[12pt]{article}
% \special{papersize=3in,5in}

\usepackage[utf8]{inputenc}
\usepackage{amssymb,amsmath,amsthm}
\pagestyle{empty}
\setlength{\parindent}{0in}

\newcommand{\detail}[1]{{\LARGE#1\par}~}
\newcommand{\refs}[1]{{\LARGE\textit{References: }#1\par}\hfill.}
\newcommand*{\abs}[1]{\left\vert#1\right\vert}
\newcommand{\lgmth}[1]{\begingroup\LARGE\[#1\]\endgroup}

%%% commands that do not need to imported into Anki:
\usepackage{mdframed}
\newcommand*{\tags}[1]{\paragraph{tags: }#1\bigskip}
\newcommand*{\xfield}[1]{\begin{mdframed}[font=\sffamily\LARGE]\centering #1\end{mdframed}\bigskip}
\newenvironment{field}{}{}
\newcommand*{\xplain}[1]{\begin{mdframed}\texttt{#1}\end{mdframed}\bigskip}
\newenvironment{plain}{\ttfamily}{\par}
\newenvironment{note}{}{}
% END OF THE PREAMBLE

% Integral dx
\newcommand{\dx}{\mathrm{d}x}

%
% Various Helper Commands
%

% For derivatives
\newcommand{\deriv}[1]{\frac{\mathrm{d}}{\mathrm{d}x} (#1)}

% For partial derivatives
\newcommand{\pderiv}[2]{\frac{\partial}{\partial #1} (#2)}


% Alias for the Solution section header
\newcommand{\solution}{\textbf{\large Solution}}

% Probability commands: Expectation, Variance, Covariance, Bias
\newcommand{\E}{\mathrm{E}}
\newcommand{\Var}{\mathrm{Var}}
\newcommand{\Cov}{\mathrm{Cov}}
\newcommand{\Bias}{\mathrm{Bias}}

% Formatting commands:

\newcommand{\mt}[1]{\ensuremath{#1}}
\newcommand{\nm}[1]{\textrm{#1}}

\newcommand\bsc[2][\DefaultOpt]{%
  \def\DefaultOpt{#2}%
  \section[#1]{#2}%
}
\newcommand\ssc[2][\DefaultOpt]{%
  \def\DefaultOpt{#2}%
  \subsection[#1]{#2}%
}
\newcommand{\bgpf}{\begin{proof} $ $\newline}

\newcommand{\bgeq}{\begin{equation*}}
\newcommand{\eeq}{\end{equation*}}	

\newcommand{\balist}{\begin{enumerate}[label=\alph*.]}
\newcommand{\elist}{\end{enumerate}}

\newcommand{\bilist}{\begin{enumerate}[label=\roman*)]}	

\newcommand{\bgsp}{\begin{split}}
% \newcommand{\esp}{\end{split}} % doesn't work for some reason.

\newcommand\prs[1]{~~~\textbf{(#1)}}

\newcommand{\lt}[1]{\textbf{Let: } #1}
     							   %  if you're setting it to be true
\newcommand{\supp}[1]{\textbf{Suppose: } #1}
     							   %  Suppose (if it'll end up false)
\newcommand{\wts}[1]{\textbf{Want to show: } #1}
     							   %  Want to show
\newcommand{\as}[1]{\textbf{Assume: } #1}
     							   %  if you think it follows from truth
\newcommand{\bpth}[1]{\textbf{(#1)}}

\newcommand{\step}[2]{\begin{equation}\tag{#2}#1\end{equation}}
\newcommand{\epf}{\end{proof}}

\newcommand{\dbs}[3]{\mt{#1_{#2_#3}}}

\newcommand{\sidenote}[1]{-----------------------------------------------------------------Side Note----------------------------------------------------------------
#1 \

---------------------------------------------------------------------------------------------------------------------------------------------}

% Analysis / Logical commands:

\newcommand{\br}{\mt{\mathbb{R}} }       % |R
\newcommand{\bq}{\mt{\mathbb{Q}} }       % |Q
\newcommand{\bn}{\mt{\mathbb{N}} }       % |N
\newcommand{\bc}{\mt{\mathbb{C}} }       % |C
\newcommand{\bz}{\mt{\mathbb{Z}} }       % |Z
\newcommand{\bi}{\bnm{\mathbb{R}}{\mathbb{Q}}} % |Irrationals 

\newcommand{\ep}{\mt{\epsilon} }         % epsilon
\newcommand{\fa}{\mt{\forall} }          % for all
\newcommand{\afa}{\mt{\alpha} }
\newcommand{\bta}{\mt{\beta} }
\newcommand{\dta}{\mt{\delta} }
\newcommand{\mem}{\mt{\in} }
\newcommand{\exs}{\mt{\exists} }

\newcommand{\es}{\mt{\emptyset} }        % empty set
\newcommand{\sbs}{\mt{\subset} }         % subset of
\newcommand{\fs}[2]{\{\uw{#1}{1}, \uw{#1}{2}, ... \uw{#1}{#2}\}}

\newcommand{\lra}{ \mt{\longrightarrow} } % implies ----->
\newcommand{\rar}{ \mt{\Rightarrow} }     % implies -->
\newcommand{\lba}{ \mt{\longmapsto} }     % element maps to |--->

\newcommand{\lla}{ \mt{\longleftarrow} }  % implies <-----
\newcommand{\lar}{ \mt{\Leftarrow} }      % implies <--

\newcommand{\av}[1]{\mt{|}#1\mt{|}}  % absolute value

\newcommand{\prn}[1]{(#1)}
\newcommand{\bk}[1]{\{#1\}}
\newcommand{\abk}[1]{\mt{\langle}#1\mt{\rangle}}

\newcommand{\ps}{\mt{+} }
\newcommand{\ms}{\mt{-} }

\newcommand{\ls}{\mt{<} }
\newcommand{\gr}{\mt{>} }

\newcommand{\lse}{\mt{\leq} }
\newcommand{\gre}{\mt{\geq} }

\newcommand{\eql}{\mt{=} }

\newcommand{\pr}{\mt{^\prime} } 		   % prime (i.e. R')
\newcommand{\uw}[2]{#1\mt{_{#2}}}
\newcommand{\uf}[2]{#1\mt{^{#2}}}
\newcommand{\frc}[2]{\mt{\frac{#1}{#2}}}
\newcommand{\lmti}[1]{\mt{\displaystyle{\lim_{#1 \to \infty}}}}
\newcommand{\limt}[2]{\mt{\displaystyle{\lim_{#1 \to #2}}}}

\newcommand{\bnm}[2]{\mt{#1\setminus{#2}}}
\newcommand{\bnt}[2]{\mt{\textrm{#1}\setminus{\textrm{#2}}}}

\newcommand{\urng}[2]{\mt{\bigcup_{#1}^{#2}}}
\newcommand{\nrng}[2]{\mt{\bigcap_{#1}^{#2}}}
\newcommand{\nck}[2]{\mt{{#1 \choose #2}}}

     							   
\newcommand{\eqn}[1]{\[#1\]}
\newcommand{\splt}[1]{\begin{split}#1\end{split}}

\newcommand{\infy}{\mt{\infty} }
\newcommand{\unn}{\mt{\cup} }
\newcommand{\inn}{\mt{\cap} }
\newcommand\tab[1][1cm]{\hspace*{#1}}
\newcommand{\rln}{ \mt{\sim} }
\newcommand{\dvd}{ \mt{\vert} }
\newcommand{\ndvd}{ \mt{\not\vert} }
\newcommand{\eqw}{ \mt{ \equiv } }
\newcommand{\lcg}{ \mt{\gamma} }

\newcommand{\edp}{\mt{\bigoplus} }

\newcommand{\wit}[1]{\mt{\widetilde{#1}}}

\newcommand{\mxc}[5]{ % Matrix Column: entry entry entry entry DIMENSION
\underset{#5 \times 1}{
  \begin{bmatrix}
     #1 \\
     #2 \\
     #3 \\
     #4
  \end{bmatrix}
  }
}

\newcommand{\mxr}[5]{ % Matrix Row:    entry entry entry entry DIMENSION
\underset{1 \times #5}{
  \begin{bmatrix}
     #1 & #2 & #3 & #4
  \end{bmatrix}
  }
}

\begin{document}

Joshua Mitchell

CS 7312 Assignment 3

\

1. The following database has 5 transactions. Let min sup \eql 60 percent (3/5) and min conf \eql 80 (4/5) percent. 

\begin{displaymath}
  \begin{bmatrix}
     \bk{m, o, n, k, e, y}  \\
     \bk{d, o, n, k, e, y} \\
     \bk{m, a, k, e} \\
     \bk{m, u, c, k, y}\\
     \bk{c, o, k, i, e} 
\end{bmatrix}
\end{displaymath}

\begin{itemize}
  \item (40 points). Run the Apriori algorithm and list Ci (candidate itemset) and Li (frequent itemset) for each iteration i. For Ci, please show the result after applying self-joining and pruning.
  
  \item List all the strong association rules in the format of \bk{item 1, item 2} \rar \bk{item 3}.
  
\end{itemize}

\uw{C}{1} \rar \uw{L}{1} \rar \uw{C}{2} \rar \uw{L}{2} \rar \uw{C}{3} \rar \uw{L}{3}:
\begin{displaymath}
\begin{tabular}{l|cc}
  set & sup & \\
  \hline
  m & 3 & \\
  o & 3 &\\
  n & 2 &\\
  k & 5 &\\
  e & 4 &\\
  y & 3 &\\
  d & 1 &\\
  a & 1 &\\
  u & 1 &\\
  c & 2 &\\
  i & 1 &
\end{tabular} \rar 
\begin{tabular}{l|cc}
  set & sup & \\
  \hline
  m & 3 & \\
  o & 3 &\\
  k & 5 &\\
  e & 4 &\\
  y & 3 &
\end{tabular} \rar 
\begin{tabular}{l|cc}
  set & sup & \\
  \hline
  mo & 1 & \\
  mk & 3 &\\
  me & 2 &\\
  my & 2 & \\
  ok & 3 & \\
  oe & 3 & \\
  oy & 2 &\\
  ke & 4 &\\
  ky & 3 &\\
  ey & 2 &
\end{tabular} \rar
\begin{tabular}{l|cc}
  set & sup & \\
  \hline
  mk & 3 &\\
  ok & 3 & \\
  oe & 3 & \\
  ke & 4 &\\
  ky & 3 &
\end{tabular} \rar
\begin{tabular}{l|cc}
  set & sup & \\
  \hline
  mko & 1 &\\
  mke & 2 & \\
  mky & 2 & \\
  oke & 3 &\\
  oky & 2 &\\
  oey & 2 &\\
  key & 2 &\\
  kmo & 1 &
\end{tabular} \rar 
\begin{tabular}{l|cc}
  set & sup & \\
  \hline
  oke & 3
\end{tabular}
\end{displaymath}

\

Strong association rules:

o \rar k, e : 60\% sup, 100\% conf

o, e \rar k : 60\% sup, 100\% conf

o, k \rar e : 60\% sup, 100\% conf

2. For a given dataset with min sup \eql 8 (absolute support), the closed patterns are \bk{a,b,c,d} with support of 9, \bk{a,b,c} with support of 11, and \bk{a, b,d} with support of 13.

\begin{itemize}
  \item (10 points). List all the max-patterns.
  \item (30 points). List all the frequent patterns together with their absolute support values.
\end{itemize}

\begin{tabular}{l|ccc}
  Pattern & Max? & Frequent? & Absolute Support Value \\
  \hline
  abcd & Yes & Yes & 9 \\
  abc & No & Yes & 11 \\
  abd & No & Yes & 13 \\
  bcd & No & Yes & 9 \\
  acd & No & Yes & 9 \\
  ab & No & Yes & 13 \\
  ac & No & Yes & 11 \\
  ad & No & Yes & 13 \\
  bc & No & Yes & 11 \\
  bd & No & Yes & 13 \\ 
  cd & No & Yes & 9 \\
  a & No & Yes & 13 \\
  b & No & Yes & 13 \\
  c & No & Yes & 11 \\ 
  d & No & Yes & 13
\end{tabular}

\end{document}