% -*- coding:utf-8 -*-
% LATEX PREAMBLE --- needs to be imported manually
\documentclass[12pt]{article}
% \special{papersize=3in,5in}

\usepackage[utf8]{inputenc}
\usepackage{amssymb,amsmath,amsthm}
\pagestyle{empty}
\setlength{\parindent}{0in}

\newcommand{\detail}[1]{{\LARGE#1\par}~}
\newcommand{\refs}[1]{{\LARGE\textit{References: }#1\par}\hfill.}
\newcommand*{\abs}[1]{\left\vert#1\right\vert}
\newcommand{\lgmth}[1]{\begingroup\LARGE\[#1\]\endgroup}

%%% commands that do not need to imported into Anki:
\usepackage{mdframed}
\newcommand*{\tags}[1]{\paragraph{tags: }#1\bigskip}
\newcommand*{\xfield}[1]{\begin{mdframed}[font=\sffamily\LARGE]\centering #1\end{mdframed}\bigskip}
\newenvironment{field}{}{}
\newcommand*{\xplain}[1]{\begin{mdframed}\texttt{#1}\end{mdframed}\bigskip}
\newenvironment{plain}{\ttfamily}{\par}
\newenvironment{note}{}{}
% END OF THE PREAMBLE

% Integral dx
\newcommand{\dx}{\mathrm{d}x}

%
% Various Helper Commands
%

% For derivatives
\newcommand{\deriv}[1]{\frac{\mathrm{d}}{\mathrm{d}x} (#1)}

% For partial derivatives
\newcommand{\pderiv}[2]{\frac{\partial}{\partial #1} (#2)}


% Alias for the Solution section header
\newcommand{\solution}{\textbf{\large Solution}}

% Probability commands: Expectation, Variance, Covariance, Bias
\newcommand{\E}{\mathrm{E}}
\newcommand{\Var}{\mathrm{Var}}
\newcommand{\Cov}{\mathrm{Cov}}
\newcommand{\Bias}{\mathrm{Bias}}

% Formatting commands:

\newcommand{\mt}[1]{\ensuremath{#1}}
\newcommand{\nm}[1]{\textrm{#1}}

\newcommand\bsc[2][\DefaultOpt]{%
  \def\DefaultOpt{#2}%
  \section[#1]{#2}%
}
\newcommand\ssc[2][\DefaultOpt]{%
  \def\DefaultOpt{#2}%
  \subsection[#1]{#2}%
}
\newcommand{\bgpf}{\begin{proof} $ $\newline}

\newcommand{\bgeq}{\begin{equation*}}
\newcommand{\eeq}{\end{equation*}}	

\newcommand{\balist}{\begin{enumerate}[label=\alph*.]}
\newcommand{\elist}{\end{enumerate}}

\newcommand{\bilist}{\begin{enumerate}[label=\roman*)]}	

\newcommand{\bgsp}{\begin{split}}
% \newcommand{\esp}{\end{split}} % doesn't work for some reason.

\newcommand\prs[1]{~~~\textbf{(#1)}}

\newcommand{\lt}[1]{\textbf{Let: } #1}
     							   %  if you're setting it to be true
\newcommand{\supp}[1]{\textbf{Suppose: } #1}
     							   %  Suppose (if it'll end up false)
\newcommand{\wts}[1]{\textbf{Want to show: } #1}
     							   %  Want to show
\newcommand{\as}[1]{\textbf{Assume: } #1}
     							   %  if you think it follows from truth
\newcommand{\bpth}[1]{\textbf{(#1)}}

\newcommand{\step}[2]{\begin{equation}\tag{#2}#1\end{equation}}
\newcommand{\epf}{\end{proof}}

\newcommand{\dbs}[3]{\mt{#1_{#2_#3}}}

\newcommand{\sidenote}[1]{-----------------------------------------------------------------Side Note----------------------------------------------------------------
#1 \

---------------------------------------------------------------------------------------------------------------------------------------------}

% Analysis / Logical commands:

\newcommand{\br}{\mt{\mathbb{R}} }       % |R
\newcommand{\bq}{\mt{\mathbb{Q}} }       % |Q
\newcommand{\bn}{\mt{\mathbb{N}} }       % |N
\newcommand{\bc}{\mt{\mathbb{C}} }       % |C
\newcommand{\bz}{\mt{\mathbb{Z}} }       % |Z
\newcommand{\bi}{\bnm{\mathbb{R}}{\mathbb{Q}}} % |Irrationals 

\newcommand{\ep}{\mt{\epsilon} }         % epsilon
\newcommand{\fa}{\mt{\forall} }          % for all
\newcommand{\afa}{\mt{\alpha} }
\newcommand{\bta}{\mt{\beta} }
\newcommand{\dta}{\mt{\delta} }
\newcommand{\mem}{\mt{\in} }
\newcommand{\exs}{\mt{\exists} }

\newcommand{\es}{\mt{\emptyset} }        % empty set
\newcommand{\sbs}{\mt{\subset} }         % subset of
\newcommand{\fs}[2]{\{\uw{#1}{1}, \uw{#1}{2}, ... \uw{#1}{#2}\}}

\newcommand{\lra}{ \mt{\longrightarrow} } % implies ----->
\newcommand{\rar}{ \mt{\Rightarrow} }     % implies -->
\newcommand{\lba}{ \mt{\longmapsto} }     % element maps to |--->

\newcommand{\lla}{ \mt{\longleftarrow} }  % implies <-----
\newcommand{\lar}{ \mt{\Leftarrow} }      % implies <--

\newcommand{\av}[1]{\mt{|}#1\mt{|}}  % absolute value

\newcommand{\prn}[1]{(#1)}
\newcommand{\bk}[1]{\{#1\}}
\newcommand{\abk}[1]{\mt{\langle}#1\mt{\rangle}}

\newcommand{\ps}{\mt{+} }
\newcommand{\ms}{\mt{-} }

\newcommand{\ls}{\mt{<} }
\newcommand{\gr}{\mt{>} }

\newcommand{\lse}{\mt{\leq} }
\newcommand{\gre}{\mt{\geq} }

\newcommand{\eql}{\mt{=} }

\newcommand{\pr}{\mt{^\prime} } 		   % prime (i.e. R')
\newcommand{\uw}[2]{#1\mt{_{#2}}}
\newcommand{\uf}[2]{#1\mt{^{#2}}}
\newcommand{\frc}[2]{\mt{\frac{#1}{#2}}}
\newcommand{\lmti}[1]{\mt{\displaystyle{\lim_{#1 \to \infty}}}}
\newcommand{\limt}[2]{\mt{\displaystyle{\lim_{#1 \to #2}}}}

\newcommand{\bnm}[2]{\mt{#1\setminus{#2}}}
\newcommand{\bnt}[2]{\mt{\textrm{#1}\setminus{\textrm{#2}}}}

\newcommand{\urng}[2]{\mt{\bigcup_{#1}^{#2}}}
\newcommand{\nrng}[2]{\mt{\bigcap_{#1}^{#2}}}
\newcommand{\nck}[2]{\mt{{#1 \choose #2}}}

     							   
\newcommand{\eqn}[1]{\[#1\]}
\newcommand{\splt}[1]{\begin{split}#1\end{split}}

\newcommand{\infy}{\mt{\infty} }
\newcommand{\unn}{\mt{\cup} }
\newcommand{\inn}{\mt{\cap} }
\newcommand\tab[1][1cm]{\hspace*{#1}}
\newcommand{\rln}{ \mt{\sim} }
\newcommand{\dvd}{ \mt{\vert} }
\newcommand{\ndvd}{ \mt{\not\vert} }
\newcommand{\eqw}{ \mt{ \equiv } }
\newcommand{\lcg}{ \mt{\gamma} }

\newcommand{\edp}{\mt{\bigoplus} }

\newcommand{\wit}[1]{\mt{\widetilde{#1}}}

\newcommand{\mxc}[5]{ % Matrix Column: entry entry entry entry DIMENSION
\underset{#5 \times 1}{
  \begin{bmatrix}
     #1 \\
     #2 \\
     #3 \\
     #4
  \end{bmatrix}
  }
}

\newcommand{\mxr}[5]{ % Matrix Row:    entry entry entry entry DIMENSION
\underset{1 \times #5}{
  \begin{bmatrix}
     #1 & #2 & #3 & #4
  \end{bmatrix}
  }
}

\begin{document}

Joshua Mitchell; MATH 5374; HW 3; 4.4 and 5.4

\

4.4

\

Two matrices A, B \mem \uf{\bc}{m \times m} are unitarily equivalent if A \eql QB\uf{Q}{*} for some unitary Q \mem \uf{\bc}{m \times m}. Is it true or false that A and B are unitarily equivalent if and only if they have the same singular values?

\

\rar 

Assume that A and B are unitarily equivalent. Specifically,

A \eql QB\uf{Q}{*} for some unitary Q \mem \uf{\bc}{m \times m}

Notice:

\begin{displaymath}
  \splt{A & = QBQ^* \\
  U_A \Sigma_A V^*_A & = Q U_B \Sigma_B V_B^*Q^* \\
  U_A \Sigma_A V^*_A & = Q U_B \tab \Sigma_B \tab V_B^*Q^* \\
  \Sigma_A & = U_A^* Q U_B \tab \Sigma_B \tab V_B^*Q^* V_A \\
  }
\end{displaymath}

Since Q, \uw{U}{B}, \uf{\uw{V}{B}}{*}, \uf{Q}{*}, \uw{U}{A}, and \uf{\uw{V}{A}}{*} are all unitary, any matrix that is a product of these 6 matrices is also unitary. Thus, no scaling is done by any product subset of these matrices.


\

Since both \uw{$\Sigma$}{A} and \uw{$\Sigma$}{B} are diagonal matrices composed of singular values of A and B, respectively, and can be written as non-scaling transformations of each other, they must have the same set of singular values.

\

\lar

Assume A and B have the same set of singular values. Specifically,

\begin{displaymath}
  A = U_A \Sigma V^*_A \tab B = U_B \Sigma V^*_B
\end{displaymath}

Then,

\begin{displaymath}
\splt{
A & = U_A \Sigma V^*_A \\
U^*_A A V_A & = \Sigma
}
\end{displaymath}

and

\begin{displaymath}
\splt{
B & = U_B U^*_A A V_A V^*_B \\
 & = U_B U^*_A \tab A \tab V_A V^*_B
}
\end{displaymath}

Now, we wish to show:

\begin{displaymath}
  U_B U^*_A = (V_A V^*_B)^{-1} = (V_A V^*_B)^* = V_B V_A^*
\end{displaymath}


\

5.4

\

Suppose A \mem \uf{\bc}{m \times m} has an SVD A \eql U$\Sigma$\uf{V}{*}. Find an eigenvalue decomposition (5.1) of the 2m x 2m hermitian matrix B:

\begin{displaymath}
  B = \underset{2m \times 2m}{
\begin{bmatrix}
   0 & A^* \\
   A & 0 
  \end{bmatrix}
}
\end{displaymath}

An eigenvalue decomposition of B will be of the form:

\begin{displaymath}
  B = XVX^{-1}
\end{displaymath}

Notice:

\begin{displaymath}
\splt{
  B & = \underset{2m \times 2m}{
\begin{bmatrix}
   0 & A^* \\
   A & 0 
  \end{bmatrix}
} \\
& = \underset{2m \times 2m}{
\begin{bmatrix}
   0 & (U\Sigma V^*)^* \\
   U\Sigma V^* & 0 
  \end{bmatrix}
} \\ 
& = \underset{2m \times 2m}{
\begin{bmatrix}
   0 & V \Sigma U^* \\
   U\Sigma V^* & 0 
  \end{bmatrix}
} \\
& = \underset{2m \times 2m}{
\begin{bmatrix}
   V & 0 \\
   0 & U
  \end{bmatrix}
}
\underset{2m \times 2m}{
\begin{bmatrix}
   0 & \Sigma \\
   \Sigma & 0 
  \end{bmatrix}
}
\underset{2m \times 2m}{
\begin{bmatrix}
   V^* & 0\\
   0 & U^* 
  \end{bmatrix}
} \\ 
& = \underset{2m \times 2m}{
\begin{bmatrix}
   0 & V \\
   U & 0
  \end{bmatrix}
}
\underset{2m \times 2m}{
\begin{bmatrix}
   \Sigma & 0 \\
   0 & \Sigma  
  \end{bmatrix}
}
\underset{2m \times 2m}{
\begin{bmatrix}
   V^* & 0\\
   0 & U^* 
  \end{bmatrix}
} \\ 
}
\end{displaymath}

The last decomposition has the singular values in the right place, but the before and after aren't inverses. The 2nd to last has them as inverses, but no decomposition in the right place.

If you have two matrices on each side of a diagonal matrix that are inverses, does this mean that the left matrix is a matrix of the eigenvectors of the  resulting matrix?


\end{document}