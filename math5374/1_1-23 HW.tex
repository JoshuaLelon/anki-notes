% -*- coding:utf-8 -*-
% LATEX PREAMBLE --- needs to be imported manually
\documentclass[12pt]{article}
% \special{papersize=3in,5in}

\usepackage[utf8]{inputenc}
\usepackage{amssymb,amsmath,amsthm}
\pagestyle{empty}
\setlength{\parindent}{0in}

\newcommand{\detail}[1]{{\LARGE#1\par}~}
\newcommand{\refs}[1]{{\LARGE\textit{References: }#1\par}\hfill.}
\newcommand*{\abs}[1]{\left\vert#1\right\vert}
\newcommand{\lgmth}[1]{\begingroup\LARGE\[#1\]\endgroup}

%%% commands that do not need to imported into Anki:
\usepackage{mdframed}
\newcommand*{\tags}[1]{\paragraph{tags: }#1\bigskip}
\newcommand*{\xfield}[1]{\begin{mdframed}[font=\sffamily\LARGE]\centering #1\end{mdframed}\bigskip}
\newenvironment{field}{}{}
\newcommand*{\xplain}[1]{\begin{mdframed}\texttt{#1}\end{mdframed}\bigskip}
\newenvironment{plain}{\ttfamily}{\par}
\newenvironment{note}{}{}
% END OF THE PREAMBLE

% Integral dx
\newcommand{\dx}{\mathrm{d}x}

%
% Various Helper Commands
%

% For derivatives
\newcommand{\deriv}[1]{\frac{\mathrm{d}}{\mathrm{d}x} (#1)}

% For partial derivatives
\newcommand{\pderiv}[2]{\frac{\partial}{\partial #1} (#2)}


% Alias for the Solution section header
\newcommand{\solution}{\textbf{\large Solution}}

% Probability commands: Expectation, Variance, Covariance, Bias
\newcommand{\E}{\mathrm{E}}
\newcommand{\Var}{\mathrm{Var}}
\newcommand{\Cov}{\mathrm{Cov}}
\newcommand{\Bias}{\mathrm{Bias}}

% Formatting commands:

\newcommand{\mt}[1]{\ensuremath{#1}}
\newcommand{\nm}[1]{\textrm{#1}}

\newcommand\bsc[2][\DefaultOpt]{%
  \def\DefaultOpt{#2}%
  \section[#1]{#2}%
}
\newcommand\ssc[2][\DefaultOpt]{%
  \def\DefaultOpt{#2}%
  \subsection[#1]{#2}%
}
\newcommand{\bgpf}{\begin{proof} $ $\newline}

\newcommand{\bgeq}{\begin{equation*}}
\newcommand{\eeq}{\end{equation*}}	

\newcommand{\balist}{\begin{enumerate}[label=\alph*.]}
\newcommand{\elist}{\end{enumerate}}

\newcommand{\bilist}{\begin{enumerate}[label=\roman*)]}	

\newcommand{\bgsp}{\begin{split}}
% \newcommand{\esp}{\end{split}} % doesn't work for some reason.

\newcommand\prs[1]{~~~\textbf{(#1)}}

\newcommand{\lt}[1]{\textbf{Let: } #1}
     							   %  if you're setting it to be true
\newcommand{\supp}[1]{\textbf{Suppose: } #1}
     							   %  Suppose (if it'll end up false)
\newcommand{\wts}[1]{\textbf{Want to show: } #1}
     							   %  Want to show
\newcommand{\as}[1]{\textbf{Assume: } #1}
     							   %  if you think it follows from truth
\newcommand{\bpth}[1]{\textbf{(#1)}}

\newcommand{\step}[2]{\begin{equation}\tag{#2}#1\end{equation}}
\newcommand{\epf}{\end{proof}}

\newcommand{\dbs}[3]{\mt{#1_{#2_#3}}}

\newcommand{\sidenote}[1]{-----------------------------------------------------------------Side Note----------------------------------------------------------------
#1 \

---------------------------------------------------------------------------------------------------------------------------------------------}

% Analysis / Logical commands:

\newcommand{\br}{\mt{\mathbb{R}} }       % |R
\newcommand{\bq}{\mt{\mathbb{Q}} }       % |Q
\newcommand{\bn}{\mt{\mathbb{N}} }       % |N
\newcommand{\bc}{\mt{\mathbb{C}} }       % |C
\newcommand{\bz}{\mt{\mathbb{Z}} }       % |Z
\newcommand{\bi}{\bnm{\mathbb{R}}{\mathbb{Q}}} % |Irrationals 

\newcommand{\ep}{\mt{\epsilon} }         % epsilon
\newcommand{\fa}{\mt{\forall} }          % for all
\newcommand{\afa}{\mt{\alpha} }
\newcommand{\bta}{\mt{\beta} }
\newcommand{\dta}{\mt{\delta} }
\newcommand{\mem}{\mt{\in} }
\newcommand{\exs}{\mt{\exists} }

\newcommand{\es}{\mt{\emptyset} }        % empty set
\newcommand{\sbs}{\mt{\subset} }         % subset of
\newcommand{\fs}[2]{\{\uw{#1}{1}, \uw{#1}{2}, ... \uw{#1}{#2}\}}

\newcommand{\lra}{ \mt{\longrightarrow} } % implies ----->
\newcommand{\rar}{ \mt{\Rightarrow} }     % implies -->
\newcommand{\lba}{ \mt{\longmapsto} }     % element maps to |--->

\newcommand{\lla}{ \mt{\longleftarrow} }  % implies <-----
\newcommand{\lar}{ \mt{\Leftarrow} }      % implies <--

\newcommand{\av}[1]{\mt{|}#1\mt{|}}  % absolute value

\newcommand{\prn}[1]{(#1)}
\newcommand{\bk}[1]{\{#1\}}
\newcommand{\abk}[1]{\mt{\langle}#1\mt{\rangle}}

\newcommand{\ps}{\mt{+} }
\newcommand{\ms}{\mt{-} }

\newcommand{\ls}{\mt{<} }
\newcommand{\gr}{\mt{>} }

\newcommand{\lse}{\mt{\leq} }
\newcommand{\gre}{\mt{\geq} }

\newcommand{\eql}{\mt{=} }

\newcommand{\pr}{\mt{^\prime} } 		   % prime (i.e. R')
\newcommand{\uw}[2]{#1\mt{_{#2}}}
\newcommand{\uf}[2]{#1\mt{^{#2}}}
\newcommand{\frc}[2]{\mt{\frac{#1}{#2}}}
\newcommand{\lmti}[1]{\mt{\displaystyle{\lim_{#1 \to \infty}}}}
\newcommand{\limt}[2]{\mt{\displaystyle{\lim_{#1 \to #2}}}}

\newcommand{\bnm}[2]{\mt{#1\setminus{#2}}}
\newcommand{\bnt}[2]{\mt{\textrm{#1}\setminus{\textrm{#2}}}}

\newcommand{\urng}[2]{\mt{\bigcup_{#1}^{#2}}}
\newcommand{\nrng}[2]{\mt{\bigcap_{#1}^{#2}}}
\newcommand{\nck}[2]{\mt{{#1 \choose #2}}}

     							   
\newcommand{\eqn}[1]{\[#1\]}
\newcommand{\splt}[1]{\begin{split}#1\end{split}}

\newcommand{\infy}{\mt{\infty} }
\newcommand{\unn}{\mt{\cup} }
\newcommand{\inn}{\mt{\cap} }
\newcommand\tab[1][1cm]{\hspace*{#1}}
\newcommand{\rln}{ \mt{\sim} }
\newcommand{\dvd}{ \mt{\vert} }
\newcommand{\ndvd}{ \mt{\not\vert} }
\newcommand{\eqw}{ \mt{ \equiv } }
\newcommand{\lcg}{ \mt{\gamma} }

\newcommand{\edp}{\mt{\bigoplus} }

\newcommand{\wit}[1]{\mt{\widetilde{#1}}}

\newcommand{\mxc}[5]{ % Matrix Column: entry entry entry entry DIMENSION
\underset{#5 \times 1}{
  \begin{bmatrix}
     #1 \\
     #2 \\
     #3 \\
     #4
  \end{bmatrix}
  }
}

\newcommand{\mxr}[5]{ % Matrix Row:    entry entry entry entry DIMENSION
\underset{1 \times #5}{
  \begin{bmatrix}
     #1 & #2 & #3 & #4
  \end{bmatrix}
  }
}

\begin{document}

01/23's HW:

\

\textbf{Exercise 1.3}: Generalizing Example 1.3, we say that a square or rectangular matrix R with entries \uw{r}{ij} is upper-triangular if \uw{r}{ij} \eql 0 for i \gr j. By considering what space is spanned by the first n columns of R and using (1.8), show that if R is a nonsingular m $\times$ m upper-triangular matrix, then \uf{R}{-1} is also upper-triangular. (The analogous result also holds for lower-triangular matrices.)

\bgpf

Let A be a nonsingular m $\times$ m upper-triangular matrix:

\begin{displaymath}
  A = \underset{m \times m}{
\begin{bmatrix}
     a_{11} & ... & ... & a_{1m} \\
     0 & ... & ... & ... \\
     ... & 0 & ... & ... \\
     0 & ... & 0 &  a_{mm}
  \end{bmatrix}
}
\end{displaymath}

Let \uw{\textbf{e}}{j} be the jth unit vector in the vector space \uf{\bc}{m}, i.e.

\begin{displaymath}
  I = \underset{m \times m}{
\begin{bmatrix}
    \vert & \vert & \vert & \vert \\
    e_1 & e_2 & ... & e_m \\
    \vert & \vert & \vert & \vert
\end{bmatrix}
}
\end{displaymath}




We want to show that Z \eql \uf{A}{-1} is also upper-triangular. Let's look at \uw{\textbf{a}}{1}, the first column of A.

\begin{displaymath}
  \textbf{e}_1 = Z\textbf{r}_1
\end{displaymath}

\begin{displaymath}
  \underset{m \times 1}{
\begin{bmatrix}
     1  \\
     0 \\
     0 \\
     0
  \end{bmatrix}
} 
= \underset{m \times m}{
\begin{bmatrix}
     z_{1,1} & ... & ... & z_{1,m} \\
     z_{2,1} & ... & ... & z_{2,m} \\
     ... & ... & ... & z_{1,m} \\
     z_{m,1} & ... & z_{m,m-1} & z_{m,m} 
  \end{bmatrix}
}
\underset{m \times 1}{
\begin{bmatrix}
     a_{1} \\
     0 \\
     ... \\
     0 
  \end{bmatrix} 
} 
\end{displaymath}

Notice:

If we look at entries 2 through m of \uw{\textbf{e}}{1}, they're all zero.

This means that \uw{\textbf{r}}{1} $\cdot$ \uw{\textbf{z}}{i} $=$ 0 for Z rows i \eql 2 through m.

Since entries 2 through m of \uw{\textbf{r}}{1} are 0,

\begin{displaymath}
  0 = z_{i,1} \times r_{11}
\end{displaymath}

for i \eql 2 through m.

\

Since \uw{r}{11} $\neq$ 0, 

\begin{displaymath}
  z_{i,1} = 0
\end{displaymath}

for i \eql 2 through m. Thus, we have:

\begin{displaymath}
  \underset{m \times 1}{
\begin{bmatrix}
     1  \\
     0 \\
     0 \\
     0
  \end{bmatrix}
} 
= \underset{m \times m}{
\begin{bmatrix}
     z_{1,1} & z_{1,2}  & ... & z_{1,m} \\
     0 & z_{2,2}  & ... & z_{2,m} \\
     ... & ...  & ... & z_{1,m} \\
     0 & z_{m,2} . & z_{m,m-1} & z_{m,m} 
  \end{bmatrix}
}
\underset{m \times 1}{
\begin{bmatrix}
     a_{1} \\
     0 \\
     ... \\
     0 
  \end{bmatrix} 
} 
\end{displaymath}

\

In the above case, Z is upper-triangular in at least the first column. We establish this as the base case, and desire to show that Z is upper triangular in general.

Inductive step:

\

Assume every entry in \uw{\textbf{z}}{j} past entry j is 0 for j \eql 1, 2, ..., i \ms 1. We want to show that it is also true for j \eql i.

Let j \eql i where i \ls m. Notice:

\begin{displaymath}
  \underset{m \times 1}{
\begin{bmatrix}
     0  \\
     ... \\
     0 \\
     1 \\
     0 \\
     ... \\
     0
  \end{bmatrix}
} 
= A\textbf{z}_j = \underset{m \times m}{
\begin{bmatrix}
     z_{1,1} & ... & ... & z_{1, j} & ... & z_{1,m} \\
     ... & ... & ... & ... & ... & ... \\
     0 & ...  & z_{j - 1, j - 1} & z_{j - 1, j} & ... & z_{j - 1, m} \\
     0 & ...  & 0 & z_{j, j} & ... & z_{j, m} \\
     0 & ...  & 0 & ... & ... & ... \\
     ... & ...  & ... & ... & ... & ... \\
     0 & ... & 0 & z_{m, j} & ... & z_{m,m} 
  \end{bmatrix}
}
\underset{m \times 1}{
\begin{bmatrix}
     a_1  \\
     ... \\
     a_{j - 1} \\
     a_j \\
     0 \\
     ... \\
     0
  \end{bmatrix} \\
}
\end{displaymath}

So,

\begin{displaymath}
A\textbf{z}_j = \underset{m \times m}{
\begin{bmatrix}
     ... & ... & ... & ... & ... & ... \\
     ... & ... & ... & ... & ... & ... \\
     ... & ...  & ... & ... & ... & ...\\
     0 & ...  & 0 & z_{j, j} & ... & z_{j, m} \\
     0 & ...  & 0 & z_{j + 1, j} & ... & z_{j + 1, m} \\
     ... & ...  & ... & ... & ... & ... \\
     0 & ... & 0 & z_{m, j} & ... & z_{m,m} 
  \end{bmatrix}
}
\underset{m \times 1}{
\begin{bmatrix}
     a_1  \\
     ... \\
     a_{j - 1} \\
     a_j \\
     0 \\
     ... \\
     0
  \end{bmatrix} \\
} 
\end{displaymath}

Notice:

To the left of the jth column of Z is all zeros, and the bottom of the a column (below \uw{a}{j}) is also all zeros. Thus, the dot product of Z's row with \uw{\textbf{a}}{j} is determined solely by the \uw{z}{k, j} entry times the \uw{a}{j} entry (for k \eql j \ps 1 to m).

\

Since 0 \eql \uw{z}{k, j} $\times$ \uw{a}{j} for k \eql j \ps 1 to m and \uw{a}{j} is non-zero by definition, \uw{z}{k, j} must be zero for k \eql j \ps 1 to m.

Hence, there is all zeros below \uw{z}{j, j} for all j.

Hence, Z \eql \uf{A}{-1} is upper-triangular.

\epf

\end{document}