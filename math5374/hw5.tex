% -*- coding:utf-8 -*-
% LATEX PREAMBLE --- needs to be imported manually
\documentclass[12pt]{article}
% \special{papersize=3in,5in}

\usepackage[utf8]{inputenc}
\usepackage{amssymb,amsmath,amsthm}
\pagestyle{empty}
\setlength{\parindent}{0in}

\newcommand{\detail}[1]{{\LARGE#1\par}~}
\newcommand{\refs}[1]{{\LARGE\textit{References: }#1\par}\hfill.}
\newcommand*{\abs}[1]{\left\vert#1\right\vert}
\newcommand{\lgmth}[1]{\begingroup\LARGE\[#1\]\endgroup}

%%% commands that do not need to imported into Anki:
\usepackage{mdframed}
\newcommand*{\tags}[1]{\paragraph{tags: }#1\bigskip}
\newcommand*{\xfield}[1]{\begin{mdframed}[font=\sffamily\LARGE]\centering #1\end{mdframed}\bigskip}
\newenvironment{field}{}{}
\newcommand*{\xplain}[1]{\begin{mdframed}\texttt{#1}\end{mdframed}\bigskip}
\newenvironment{plain}{\ttfamily}{\par}
\newenvironment{note}{}{}
% END OF THE PREAMBLE

% Integral dx
\newcommand{\dx}{\mathrm{d}x}

%
% Various Helper Commands
%

% For derivatives
\newcommand{\deriv}[1]{\frac{\mathrm{d}}{\mathrm{d}x} (#1)}

% For partial derivatives
\newcommand{\pderiv}[2]{\frac{\partial}{\partial #1} (#2)}


% Alias for the Solution section header
\newcommand{\solution}{\textbf{\large Solution}}

% Probability commands: Expectation, Variance, Covariance, Bias
\newcommand{\E}{\mathrm{E}}
\newcommand{\Var}{\mathrm{Var}}
\newcommand{\Cov}{\mathrm{Cov}}
\newcommand{\Bias}{\mathrm{Bias}}

% Formatting commands:

\newcommand{\mt}[1]{\ensuremath{#1}}
\newcommand{\nm}[1]{\textrm{#1}}

\newcommand\bsc[2][\DefaultOpt]{%
  \def\DefaultOpt{#2}%
  \section[#1]{#2}%
}
\newcommand\ssc[2][\DefaultOpt]{%
  \def\DefaultOpt{#2}%
  \subsection[#1]{#2}%
}
\newcommand{\bgpf}{\begin{proof} $ $\newline}

\newcommand{\bgeq}{\begin{equation*}}
\newcommand{\eeq}{\end{equation*}}	

\newcommand{\balist}{\begin{enumerate}[label=\alph*.]}
\newcommand{\elist}{\end{enumerate}}

\newcommand{\bilist}{\begin{enumerate}[label=\roman*)]}	

\newcommand{\bgsp}{\begin{split}}
% \newcommand{\esp}{\end{split}} % doesn't work for some reason.

\newcommand\prs[1]{~~~\textbf{(#1)}}

\newcommand{\lt}[1]{\textbf{Let: } #1}
     							   %  if you're setting it to be true
\newcommand{\supp}[1]{\textbf{Suppose: } #1}
     							   %  Suppose (if it'll end up false)
\newcommand{\wts}[1]{\textbf{Want to show: } #1}
     							   %  Want to show
\newcommand{\as}[1]{\textbf{Assume: } #1}
     							   %  if you think it follows from truth
\newcommand{\bpth}[1]{\textbf{(#1)}}

\newcommand{\step}[2]{\begin{equation}\tag{#2}#1\end{equation}}
\newcommand{\epf}{\end{proof}}

\newcommand{\dbs}[3]{\mt{#1_{#2_#3}}}

\newcommand{\sidenote}[1]{-----------------------------------------------------------------Side Note----------------------------------------------------------------
#1 \

---------------------------------------------------------------------------------------------------------------------------------------------}

% Analysis / Logical commands:

\newcommand{\br}{\mt{\mathbb{R}} }       % |R
\newcommand{\bq}{\mt{\mathbb{Q}} }       % |Q
\newcommand{\bn}{\mt{\mathbb{N}} }       % |N
\newcommand{\bc}{\mt{\mathbb{C}} }       % |C
\newcommand{\bz}{\mt{\mathbb{Z}} }       % |Z
\newcommand{\bi}{\bnm{\mathbb{R}}{\mathbb{Q}}} % |Irrationals 

\newcommand{\ep}{\mt{\epsilon} }         % epsilon
\newcommand{\fa}{\mt{\forall} }          % for all
\newcommand{\afa}{\mt{\alpha} }
\newcommand{\bta}{\mt{\beta} }
\newcommand{\dta}{\mt{\delta} }
\newcommand{\mem}{\mt{\in} }
\newcommand{\exs}{\mt{\exists} }

\newcommand{\es}{\mt{\emptyset} }        % empty set
\newcommand{\sbs}{\mt{\subset} }         % subset of
\newcommand{\fs}[2]{\{\uw{#1}{1}, \uw{#1}{2}, ... \uw{#1}{#2}\}}

\newcommand{\lra}{ \mt{\longrightarrow} } % implies ----->
\newcommand{\rar}{ \mt{\Rightarrow} }     % implies -->
\newcommand{\lba}{ \mt{\longmapsto} }     % element maps to |--->

\newcommand{\lla}{ \mt{\longleftarrow} }  % implies <-----
\newcommand{\lar}{ \mt{\Leftarrow} }      % implies <--

\newcommand{\av}[1]{\mt{|}#1\mt{|}}  % absolute value

\newcommand{\prn}[1]{(#1)}
\newcommand{\bk}[1]{\{#1\}}
\newcommand{\abk}[1]{\mt{\langle}#1\mt{\rangle}}

\newcommand{\ps}{\mt{+} }
\newcommand{\ms}{\mt{-} }

\newcommand{\ls}{\mt{<} }
\newcommand{\gr}{\mt{>} }

\newcommand{\lse}{\mt{\leq} }
\newcommand{\gre}{\mt{\geq} }

\newcommand{\eql}{\mt{=} }

\newcommand{\pr}{\mt{^\prime} } 		   % prime (i.e. R')
\newcommand{\uw}[2]{#1\mt{_{#2}}}
\newcommand{\uf}[2]{#1\mt{^{#2}}}
\newcommand{\frc}[2]{\mt{\frac{#1}{#2}}}
\newcommand{\lmti}[1]{\mt{\displaystyle{\lim_{#1 \to \infty}}}}
\newcommand{\limt}[2]{\mt{\displaystyle{\lim_{#1 \to #2}}}}

\newcommand{\bnm}[2]{\mt{#1\setminus{#2}}}
\newcommand{\bnt}[2]{\mt{\textrm{#1}\setminus{\textrm{#2}}}}

\newcommand{\urng}[2]{\mt{\bigcup_{#1}^{#2}}}
\newcommand{\nrng}[2]{\mt{\bigcap_{#1}^{#2}}}
\newcommand{\nck}[2]{\mt{{#1 \choose #2}}}

     							   
\newcommand{\eqn}[1]{\[#1\]}
\newcommand{\splt}[1]{\begin{split}#1\end{split}}

\newcommand{\infy}{\mt{\infty} }
\newcommand{\unn}{\mt{\cup} }
\newcommand{\inn}{\mt{\cap} }
\newcommand\tab[1][1cm]{\hspace*{#1}}
\newcommand{\rln}{ \mt{\sim} }
\newcommand{\dvd}{ \mt{\vert} }
\newcommand{\ndvd}{ \mt{\not\vert} }
\newcommand{\eqw}{ \mt{ \equiv } }
\newcommand{\lcg}{ \mt{\gamma} }

\newcommand{\edp}{\mt{\bigoplus} }

\newcommand{\wit}[1]{\mt{\widetilde{#1}}}

\newcommand{\mxc}[5]{ % Matrix Column: entry entry entry entry DIMENSION
\underset{#5 \times 1}{
  \begin{bmatrix}
     #1 \\
     #2 \\
     #3 \\
     #4
  \end{bmatrix}
  }
}

\newcommand{\mxr}[5]{ % Matrix Row:    entry entry entry entry DIMENSION
\underset{1 \times #5}{
  \begin{bmatrix}
     #1 & #2 & #3 & #4
  \end{bmatrix}
  }
}

\begin{document}

Joshua Mitchell; MATH 5374; HW 5; 7.5 

\

7.5

\

Let A be an m $\times$ n matrix (m \gre n), and let A \eql $\hat Q \hat R$ be a reduced QR factorization.

\begin{enumerate}
  \item Show that A has rank n if and only if all the diagonal entries of $\hat R$ are non-zero.
  
 \rar
 
 Suppose that A has rank n.
 
 Want to show: A has rank equal to the number of non-zero diagonal entries in $\hat R$ by induction.
 
 Base case: Column 1 of A spans 1 dimension. Since \uw{q}{1} is non-zero and \uw{a}{1} is non-zero, that implies that \uw{r}{1, 1} is non-zero.
 
 Inductive step: Assume that A has non-zero diagonal entries up to column k \ms 1, k \ls n.
 
 Now, we want to show that if A has non-zero diagonal entries up to column k \ms 1, the kth diagonal entry of $\hat R$ must also be non-zero.
 
 However, suppose that the kth diagonal is 0. 
 
 Since A is full rank up to column k \ms 1, that implies that its $\hat Q \hat R$ decomposition up to k \ms 1 is also full rank.
 
 This also means that any vector in \uf{\bc}{k \ms 1} can be written as a linear combination of the first k \ms 1 columns of A and of $\hat Q \hat R$.
 
 Since $\hat R_{i, k}$ for i:= k ... n is 0 and $\hat Q$ is orthonormal, $A_k$ \eql $\hat Q \hat R_k$ is a vector that lies in the span of the first k \ms 1 columns of A.
 
 This is a contradiction, since A is full rank. Hence, $\hat R_{k, k}$ must be non-zero.
 
 Thus, by induction, all diagonal entries of $\hat R$ must be non-zero.
 
 \newpage
 
 \lar
 
 Assume that all diagonal entries of $\hat R$ are non-zero.
 
 Then
 \begin{displaymath}
  \splt{a_1 & = r_{11}q_1 \\
  a_2 & = r_{12}q_1 + r_{22}q_2 \\
  & ... \\
  a_n & = r_{1n}q_1 + r_{2n}q_2 + ... + r_{nn}q_n
  }
\end{displaymath}

where \uw{r}{i, i} $\neq$ 0, for i:= 1 to n.

Notice: A up to \uw{a}{1} is rank 1 as a base case.

Assume A up to \uw{A}{k - 1} is rank k \ms 1.

Then

\begin{displaymath}
  a_k = r_{1k}q_1 + r_{2k}q_2 + ... + r_{kk}q_k
\end{displaymath}

Since A up to k \ms 1 is full rank, \uw{r}{k, k} is nonzero, and $\hat Q$ is orthonormal, the $r_{k, k}q_k$ vector is not a linear combination of $\hat Q \hat R_i$ for i:= 1 ... k \ms 1. Therefore, \uw{a}{k} is not a linear combination of the first k \ms 1 columns of A, and A up to column k has rank k.

Hence, by induction, A has rank n.

  
  \item Suppose that $\hat R$ has k nonzero diagonal entries for some k with 0 \lse k \ls n. What does this imply about the rank of A? Exactly k? At least k? At most k? Give a precise answer, and prove it.
 
 Well, if $\hat R$ has 0 non-zero diagonal entries (i.e. all zeros down the diagonal) but has non-zero entries everywhere else in the upper part, then it still has rank n \ms 1. So it can't be exactly k nor at most k. So if the only other option is at least k, then that has to be it.
 
 Proof:
 
 Suppose $\hat R$ has k non-zero diagonal entries. Then each corresponding column of A is a linear combination of orthonormal vectors from $\hat Q$. Since each successive linear combination adds a new dimension to the rank of A due to the non-zero entry on the diagonal, that means A has at least rank k. 
 
\end{enumerate}


\end{document}